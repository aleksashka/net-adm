\section[ipv4]{Layer 3 (Internet Protocol)}

\subsection{IP, IP Address and Subnet Mask}
\begin{frame}{IP, IP Address and Subnet Mask}
	\wiki{Internet Protocol}{Internet_Protocol} \wiki{Address}{IPv4\#Addressing}\pause
	\begin{itemize}[<+->]
		\item \rfc{791}
		\item 32-bit length (4 294 967 296 addresses)
		\item Divided into 4 octets (bytes)
		\item Each octet is converted from binary to decimal
		\item \texttt{00001010 00000001 00000001 01100100}
		\item 10.1.1.100
		\item Identification of any host on a network
	\end{itemize}
	\onslide<9->
	Mask\pause
	\begin{itemize}[<+->]
		\item 32-bit length
		%\item In binary: '1' is not allowed after '0'
		\item \texttt{11111111 11111111 11111111 00000000}
		\item 255.255.255.0
	\end{itemize}
\end{frame}

\subsection{Binary-Decimal-Binary conversion}
\begin{frame}{Binary-Decimal-Binary conversion}
	\textit{There are 10 types of people in the world. Those who understand binary and those who don't.}
	\pause
	\texttt{
		\begin{center}
			\begin{tabular}[t]{r}
				Decimal \\ \pause
				{[0-9]}	\\ \pause
				0	\\ \pause
				1	\\ \pause
				2	\\ \pause
				...	\\ \pause
				9	\\ \pause
				10	\\ \pause
				... \\ \pause
				99	\\ \pause
				100\\ \pause
			\end{tabular}
			\hspace{1.5cm}
			\begin{tabular}[t]{r r}
				Decimal		&	Binary	\\
				{[0-9]}		&	[0-1]	\\ \hline \pause
				0	\pause	&	0		\\ \pause
				1	\pause	&	1		\\ \pause
				2	\pause	&	10		\\ \pause
				3	\pause	&	11		\\ \pause
				4	\pause	&	100		\\ \pause
				5	\pause	&	101		\\ \pause
				6	\pause	&	110		\\ \pause
				7	\pause	&	111
			\end{tabular}
		\end{center}
	}
\end{frame}

\begin{frame}{Binary to Decimal conversion}
	\texttt{
		\begin{center}
			\begin{tabular}{*{9}c}
				2\textsuperscript{7} & 2\textsuperscript{6} & 2\textsuperscript{5} &
				2\textsuperscript{4} & 2\textsuperscript{3} & 2\textsuperscript{2} &
				2\textsuperscript{1} & 2\textsuperscript{0} \\ \pause
				128 & 64 & 32 & 16 & 8 & 4 & 2 & 1 \\ \pause
				  0 &  1 &  0 &  0 & 0 & 1 & 0 & 1 \\ \pause
				    & 64 &    &    &   & 4 &   & 1 \pause & 69 \\
			\end{tabular}
		\end{center}
	}
\end{frame}

\begin{frame}{Decimal to Binary conversion}
	\only<beamer>{\only<1>{\textit{Change analogy}}}
	\pause
	\texttt{
		\begin{center}
			\begin{tabular}{*{9}c}
				          & 128 & 64 & 32        & 16 &  8        & 4        & 2        & 1 \\
				46 \pause &     &    & 46 \pause &    & 14 \pause & 6 \pause & 2 \pause &   \\
				          &   0 &  0 &  1        &  0 &  1        & 1        & 1        & 0 \\
			\end{tabular}
		\end{center}
	}
\end{frame}

\subsection{Meaning of the Mask}
\begin{frame}{Meaning of the Mask}
	\texttt{
		\onslide<3->{\\10.1.1.100 255.255.255.0}
		\onslide<4->{\\10.1.1.100/24\vspace{0.2cm}}
		\only<9->{\\10.1.1.0\\10.1.1.1\\10.1.1.254\\10.1.1.255\vspace{0.2cm}}
		\only<1,3->{\\00001010 00000001 00000001 01100100\hspace{0.5cm}IP Address}
		\only<2|handout:0>{\\00001010 00000001 00000001\colorbox{green}{01100100}\hspace{0.5cm}IP Address}
		\onslide<2->{\\11111111 11111111 11111111 00000000\hspace{0.5cm}Mask
		\onslide<5->{\noindent\rule[0.5ex]{\linewidth}{0.7pt}}
		\onslide<5->{\\00001010 00000001 00000001 00000000\hspace{0.5cm}Network Ad.}
		\onslide<7->{\\00001010 00000001 00000001 00000001\hspace{0.5cm}First Ad.}
		\onslide<8->{\\00001010 00000001 00000001 11111110\hspace{0.5cm}Last Address}
		\onslide<6->{\\00001010 00000001 00000001 11111111\hspace{0.5cm}Broadcast}
		}
	}
\end{frame}

\subsection{Cisco Commands}
\begin{frame}{Commands: Hostname, IP Address, Save Config}
	\texttt{
		\only<2-3|handout:0>{\pause\\\textit{Press RETURN to get started}
		\vspace{0.5cm}}
		\pause\\Router>\onslide<4->{\underline{en}able} \hfill \only<3,5->{User EXEC mode}\only<4|handout:0>{Go to Privileged EXEC}
		\pause\pause\pause\\Router\#\onslide<5->{\underline{conf}igure \underline{t}erminal} \hfill Privileged EXEC mode
		\pause\\Router(config)\#hostname R\hfill Configuration mode
		\pause\\R(config)\#\pause interface GigabitEthernet0/1
		\pause\\R(config-if)\#ip address 10.1.1.1 255.255.255.0
		\pause\\R(config-if)\#no shutdown
		\pause\\R(config-if)\#exit \only<beamer>{\hfill or \underline{end} to go to \#}
		\pause\\R(config)\#exit
		\pause\\R\#show startup-config
		\pause\\R\#show running-config \pause {[ interface \{ gi0/1 \} ]}
		\pause\\R\#show ip interface [ gi0/1 | brief [ gi0/1 ] ]
		\pause\\R\#copy running-config startup-config
		\pause\\Destination filename [startup-config]?%\hfill <ENTER>
		\pause\\Building configuration...
		\\{[OK]}
		%\pause\\
	}
\end{frame}

\subsection{Types and Classes of IP Addresses}
\begin{frame}{Types and Classes of IP Addresses}
	\begin{itemize}[<+->]
		\item Localhost: Local communications only (127.0.0.0/8)
		\item Unicast: One-to-one communication (e.g. 10.1.1.1)
		\item Multicast: One-to-many communication (e.g. 224.0.0.9)
		\item Local Broadcast: One to all local hosts (255.255.255.255)
		\item Directed Broadcast: One to all hosts in remote network (e.g. 172.16.255.255)
	\end{itemize}
	%\vspace{0.5cm}
	\onslide<6->{
		\begin{center}
			\begin{tabular}{*{9}l}
				Class & First bits &        First octet & Default mask    \\ \pause
				A     & 0          & \pause 1-126       & 255.0.0.0       \\ \pause
				B     & 10         & \pause 128-191     & 255.255.0.0     \\ \pause
				C     & 110        & \pause 192-223     & 255.255.255.0   \\ \pause
				D     & 1110       & \pause 224-239     & Multicast range \\ \pause
				E     & 1111       & \pause 240-255     & Reserved
			\end{tabular}
		\end{center}
	}
\end{frame}

\subsection{IP Routing}
\begin{frame}{IP Routing}
	\begin{itemize}[<+->]
		\item \wiki{Packet switching}{Packet_switching}
		\item \texttt{show ip route} (EXEC command) shows routing table
		\item C (Connected) routes are "interface" routes
		\item Routers can only reach known networks
		\item If destination address of an incoming packet is not in the routing table, then the packet is dropped
		\item Routers should "learn" all routes they need using:
		\begin{itemize}
			\item Static routing (manual configuration):
			\begin{itemize}
				\item \texttt{)\#ip route 192.168.1.0 255.255.255.0 10.1.1.2}
			\end{itemize}
			\item Dynamic routing (automatically):
			\begin{itemize}
				\item RIP
				\item EIGRP
				\item OSPF
				\item BGP
			\end{itemize}
		\end{itemize}
	\end{itemize}
\end{frame}

\subsection{Default Gateway}
\begin{frame}{Default Gateway}
	\begin{itemize}[<+->]
		\item Communication to hosts in the same network (identical network bits) is direct
		\item Communication to hosts outside local network is via default gateway (should be in the same network)
		\item 10.1.1.100/24 $\rightarrow$ 10.1.1.200 [Direct or via DG]?
		\item 10.1.1.100/24 $\rightarrow$ 10.1.2.200 [Direct or via DG]?
		\item 10.1.1.100/24 $\rightarrow$ 10.2.1.200 [Direct or via DG]?
		\item 10.1.1.100/24 $\rightarrow$ 20.1.1.200 [Direct or via DG]?
		\item \texttt{)\#ip route 192.168.1.0 255.255.255.0 10.1.1.1}
		\item \texttt{)\#ip route 192.168.0.0 255.255.0.0~~~10.1.1.1}
		\item \texttt{)\#ip route 192.0.0.0~~~255.0.0.0~~~~~10.1.1.1}
		\item \texttt{)\#ip route 0.0.0.0~~~~~0.0.0.0~~~~~~~10.1.1.1}
		\item \texttt{)\#ip default-gateway 10.1.1.1}
	\end{itemize}
\end{frame}
