\section[ethr]{Layer 2 (Ethernet)}

\begin{frame}{LAYER 2 (ETHERNET)}
	\begin{itemize}
		\item Cabling and Connectors
		\item Collisions and duplex
		\item Ethernet frame structure
		\item Structure and types of MAC addresses
		\item Ethernet Switches
		\item Port Security
	\end{itemize}
\end{frame}

\begin{frame}{Cabling and Connectors}
	Cables:\pause
	\begin{itemize}[<+->]
		\item \wiki{Coaxial cable}{Coaxial_cable}
		\item \wiki{Twisted pair}{Twisted_pair}
		\item \wiki{Optical fiber cable}{Optical_fiber_cable}
	\end{itemize}
	\onslide<5->{\vspace{0.5cm}Connectors:}\pause
	\begin{itemize}[<+->]
		\item \wiki{Twisted pair connectors}{Modular_connector\#8P8C}
		\begin{itemize}
			\item \wiki{T568A/T568B}{TIA/EIA-568\#Wiring}
			\item Straight-through cable
			\begin{itemize}
				\item Hubs/Switches to other devices
			\end{itemize}
			\item Crossover cable
			\begin{itemize}
				\item Hubs/Switches to Hubs/Switches
				\item Other devices to other devices
			\end{itemize}
		\end{itemize}
		\item \wiki{Optical fiber connector}{Optical_fiber_connector}
	\end{itemize}
\end{frame}

\begin{frame}{Collisions and duplex}
	\wiki{Carrier-sense multiple access with collision detection (CSMA/CD, half-duplex)}{Carrier-sense_multiple_access_with_collision_detection}\pause
	\begin{itemize}[<+->]
		\item Is medium idle?
		\item Transmit a frame and monitor for collision
		\item If collision occurs:
		\begin{itemize}
			\item Send jam signal (make sure collision is detected)
			\item Wait random time (based on number of attempts)
			\item Try sending the frame again
		\end{itemize}
	\end{itemize}
	\onslide<8->{\wiki{Half-duplex}{Half-duplex}}\pause
	\begin{itemize}[<+->]
		\item CSMA/CD
		\item Hub
	\end{itemize}
	\onslide<11->{\wiki{Full-duplex}{Full-duplex}}\pause
	\begin{itemize}[<+->]
		\item \sout{CSMA/CD}
		\item Switch
	\end{itemize}
\end{frame}

\begin{frame}{Ethernet frame structure}
	\wiki{Ethernet frame}{Ethernet_frame} structure\pause
	\begin{itemize}[<+->]
		\item Preamble (7 bytes)
		\item Start of frame delimiter (1 byte)
		\item MAC destination (6 bytes)
		\item MAC source (6 bytes)
		\item Ethertype (2 bytes)
		\item Payload (46-1500 bytes)
		\item Frame check sequence (4 bytes)
		\item Interpacket gap (12 bytes)
	\end{itemize}
	\only<10-|handout:0>{Wireshark lab\pause
		\begin{itemize}[<+->]
			\item Which fields are not shown in Wireshark and why?
		\end{itemize}
	}
\end{frame}

\begin{frame}{Structure and types of MAC addresses}
	\wiki{MAC address}{MAC_address} structure\pause
	\begin{itemize}[<+->]
		\item Total length is 6 bytes (48 bits, 12 hex characters)
		\item First half is Organisationally Unique Identifier (OUI)
		\begin{itemize}
			\item 7th bit (U/L or LG bit) shows if address is universally/globally (0) or locally administered (1)
			\item 8th bit (U/M or IG bit) determines unicast/individual (0) or multicast/group (1) frame
		\end{itemize}
		\item Second half is NIC specific
	\end{itemize}
	\onslide<7->{MAC address types}\pause
	\begin{itemize}[<+->]
		\item Unicast (received by single host)
		\item Multicast (received by multiple hosts)
		\item Broadcast (received by all hosts)
	\end{itemize}
	\only<11-|handout:0>{Wireshark lab\pause
		\begin{itemize}[<+->]
			\item Find OUI, LG and IG fields
		\end{itemize}
	}
\end{frame}

\begin{frame}{Ethernet Switches}
	\begin{itemize}[<+->]
		\item Process frames based on MAC Address Table state:
		\begin{itemize}
			\item Flood (send frame from one port to all other ports) unknown unicast, multicast and broadcast
			\item Forward (send frame from one port to another) known unicast
			\item Filter (do not send anywhere) frames which do not need to be forwarded
		\end{itemize}
		\item Learn MAC Addresses (fill MAC Addres Table)
	\end{itemize}
	\onslide<6->
	\texttt{
		\\      \#show interface status
		\\\pause\#show mac address-table interface \textit{type number}
		\\\pause\#show mac address-table address \textit{mac-addr}
	}
	\\\vspace{0.3cm}
	\hspace{2cm}
	\begin{tabular}{ll}
		  \pause \texttt{Interface}	&\texttt{MAC Address}
		\\\pause \texttt{Gi0/5}		&\texttt{0123.4567.89ab}
		\\\pause \texttt{Gi0/%
		 \only<11>          {6}%
		 \only<12|handout:0>{5}}	&\texttt{0123.4567.89ac}
	\end{tabular}\\
\end{frame}

\begin{frame}{Port Security}
	Restricts input to an interface by limiting and identifying MAC addresses
	\vspace{0.3cm}
	\texttt{
		\\\pause )\#interface FastEthernet0/2
		\\\pause -if)\#\underline{sw}itchport mode access
		\\\pause -if)\#sw \underline{po}rt-security maximum \textit{maximum}
		\\\pause -if)\#sw po mac-address \textit{mac-addr}
		\\\pause -if)\#sw po mac-address sticky \pause\textit{[mac-addr]}
		\hangindent=2em{
		\\\pause -if)\#sw po violation \{ protect | restrict | shutdown \}}
		\\\pause -if)\#switchport port-security
	}
	\\\vspace{0.1cm}\hspace{1.5cm}
	\begin{tabular}{l*{3}c}
		        \pause Violation	&Drop	&Log	&Error-disable
		\\\hline\pause Protect		&Yes
		\\      \pause Restrict		&Yes	&Yes
		\\      \pause Shutdown*		&Yes	&Yes	&Yes
	\end{tabular}\\
\end{frame}

\begin{frame}{Port Security (Cont.)}
	\texttt{
		\\\#show interface [\textit{ type number }] status
		\\\pause Port~~~~~~Name~~~~~Status
		\\       Fa0/1~~~~~~~~~~~~~~connected
		\\       Fa0/2~~~~~~~~~~~~~~notconnect
		\\       Fa0/3~~~~~~~~~~~~~~disabled
		\\       Fa0/4~~~~~~~~~~~~~~err-disabled\vspace{0.3cm}
		\\\pause \#show interface status err-disabled
		\\\pause Port~~~~~~Name~~~~~Status~~~~~~~Reason
		\\       Fa0/4~~~~~~~~~~~~~~err-disabled~psecure-violation
	}
	\\\vspace{0.3cm}\pause
	To bring interface back either \texttt{shutdown/no shutdown} interface or use automatic error-disable recovery feature:
	\pause\vspace{0.3cm}
	\texttt{
		\\)\#errdisable recovery cause psecure-violation
		\\)\#errdisable recovery interval \textit{interval}
	}\\
\end{frame}
